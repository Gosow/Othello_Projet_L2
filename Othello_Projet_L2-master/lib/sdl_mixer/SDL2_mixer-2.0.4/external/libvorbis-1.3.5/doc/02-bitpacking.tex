% -*- mode: latex; TeX-master: "Vorbis_I_spec"; -*-
%!TEX root = Vorbis_I_spec.tex
% $Id$
\section{Bitpacking Convention} \label{vorbis:spec:bitpacking}

\subsection{Overview}

The Vorbis codec uses relatively unstructured raw packets containing
arbitrary-width binary integer fields.  Logically, these packets are a
bitstream in which bits are coded one-by-one by the encoder and then
read one-by-one in the same monotonically increasing order by the
decoder.  Most current binary storage arrangements group bits into a
native word size of eight bits (octets), sixteen bits, thirty-two bits
or, less commonly other fixed word sizes.  The Vorbis bitpacking
convention specifies the correct mapping of the logical packet
bitstream into an actual representation in fixed-width words.


\subsubsection{octets, bytes and words}

In most contemporary architectures, a 'byte' is synonymous with an
'octet', that is, eight bits.  This has not always been the case;
seven, ten, eleven and sixteen bit 'bytes' have been used.  For
purposes of the bitpacking convention, a byte implies the native,
smallest integer storage representation offered by a platform.  On
modern platforms, this is generally assumed to be eight bits (not
necessarily because of the processor but because of the
filesystem/memory architecture.  Modern filesystems invariably offer
bytes as the fundamental atom of storage).  A 'word' is an integer
size that is a grouped multiple of this smallest size.

The most ubiquitous architectures today consider a 'byte' to be an
octet (eight bits) and a word to be a group of two, four or eight
bytes (16, 32 or 64 bits).  Note however that the Vorbis bitpacking
convention is still well defined for any native byte size; Vorbis uses
the native bit-width of a given storage system. This document assumes
that a byte is one octet for purposes of example.

\subsubsection{bit order}

A byte has a well-defined 'least significant' bit (LSb), which is the
only bit set when the byte is storing the two's complement integer
value +1.  A byte's 'most significant' bit (MSb) is at the opposite
end of the byte. Bits in a byte are numbered from zero at the LSb to
$n$ ($n=7$ in an octet) for the
MSb.



\subsubsection{byte order}

Words are native groupings of multiple bytes.  Several byte orderings
are possible in a word; the common ones are 3-2-1-0 ('big endian' or
'most significant byte first' in which the highest-valued byte comes
first), 0-1-2-3 ('little endian' or 'least significant byte first' in
which the lowest value byte comes first) and less commonly 3-1-2-0 and
0-2-1-3 ('mixed endian').

The Vorbis bitpacking convention specifies storage and bitstream
manipulation at the byte, not word, level, thus host word ordering is
of a concern only during optimization when writing high performance
code that operates on a word of storage at a time rather than by byte.
Logically, bytes are always coded and decoded in order from byte zero
through byte $n$.



\subsubsection{coding bits into byte sequences}

The Vorbis codec has need to code arbitrary bit-width integers, from
zero to 32 bits wide, into packets.  These integer fields are not
aligned to the boundaries of the byte representation; the next field
is written at the bit position at which the previous field ends.

The encoder logically packs integers by writing the LSb of a binary
integer to the logical bitstream first, followed by next least
significant bit, etc, until the requested number of bits have been
coded.  When packing the bits into bytes, the encoder begins by
placing the LSb of the integer to be written into the least
significant unused bit position of the destination byte, followed by
the next-least significant bit of the source integer and so on up to
the requested number of bits.  When all bits of the destination byte
have been filled, encoding continues by zeroing all bits of the next
byte and writing the next bit into the bit position 0 of that byte.
Decoding follows the same process as encoding, but by reading bits
from the byte stream and reassembling them into integers.



\subsubsection{signedness}

The signedness of a specific number resulting from decode is to be
interpreted by the decoder given decode context.  That is, the three
bit binary pattern 'b111' can be taken to represent either 'seven' as
an unsigned integer, or '-1' as a signed, two's complement integer.
The encoder and decoder are responsible for knowing if fields are to
be treated as signed or unsigned.



\subsubsection{coding example}

Code the 4 bit integer value '12' [b1100] into an empty bytestream.
Bytestream result:

\begin{Verbatim}[commandchars=\\\{\}]
              |
              V

        7 6 5 4 3 2 1 0
byte 0 [0 0 0 0 1 1 0 0]  <-
byte 1 [               ]
byte 2 [               ]
byte 3 [               ]
             ...
byte n [               ]  bytestream length == 1 byte

\end{Verbatim}


Continue by coding the 3 bit integer value '-1' [b111]:

\begin{Verbatim}[commandchars=\\\{\}]
        |
        V

        7 6 5 4 3 2 1 0
byte 0 [0 1 1 1 1 1 0 0]  <-
byte 1 [               ]
byte 2 [               ]
byte 3 [               ]
             ...
byte n [               ]  bytestream length == 1 byte
\end{Verbatim}


Continue by coding the 7 bit integer value '17' [b0010001]:

\begin{Verbatim}[commandchars=\\\{\}]
          |
          V

        7 6 5 4 3 2 1 0
byte 0 [1 1 1 1 1 1 0 0]
byte 1 [0 0 0 0 1 0 0 0]  <-
byte 2 [               ]
byte 3 [               ]
             ...
byte n [               ]  bytestream length == 2 bytes
                          bit cursor == 6
\end{Verbatim}


Continue by coding the 13 bit integer value '6969' [b110 11001110 01]:

\begin{Verbatim}[commandchars=\\\{\}]
                |
                V

        7 6 5 4 3 2 1 0
byte 0 [1 1 1 1 1 1 0 0]
byte 1 [0 1 0 0 1 0 0 0]
byte 2 [1 1 0 0 1 1 1 0]
byte 3 [0 0 0 0 0 1 1 0]  <-
             ...
byte n [               ]  bytestream length == 4 bytes

\end{Verbatim}




\subsubsection{decoding example}

Reading from the beginning of the bytestream encoded in the above example:

\begin{Verbatim}[commandchars=\\\{\}]
                      |
                      V

        7 6 5 4 3 2 1 0
byte 0 [1 1 1 1 1 1 0 0]  <-
byte 1 [0 1 0 0 1 0 0 0]
byte 2 [1 1 0 0 1 1 1 0]
byte 3 [0 0 0 0 0 1 1 0]  bytestream length == 4 bytes

\end{Verbatim}


We read two, two-bit integer fields, resulting in the returned numbers
'b00' and 'b11'.  Two things are worth noting here:

\begin{itemize}
\item Although these four bits were originally written as a single
four-bit integer, reading some other combination of bit-widths from the
bitstream is well defined.  There are no artificial alignment
boundaries maintained in the bitstream.

\item The second value is the
two-bit-wide integer 'b11'.  This value may be interpreted either as
the unsigned value '3', or the signed value '-1'.  Signedness is
dependent on decode context.
\end{itemize}




\subsubsection{end-of-packet alignment}

The typical use of bitpacking is to produce many independent
byte-aligned packets which are embedded into a larger byte-aligned
container structure, such as an Ogg transport bitstream.  Externally,
each bytestream (encoded bitstream) must begin and end on a byte
boundary.  Often, the encoded bitstream is not an integer number of
bytes, and so there is unused (uncoded) space in the last byte of a
packet.

Unused space in the last byte of a bytestream is always zeroed during
the coding process.  Thus, should this unused space be read, it will
return binary zeroes.

Attempting to read past the end of an encoded packet results in an
'end-of-packet' condition.  End-of-packet is not to be considered an
error; it is merely a state indicating that there is insufficient
remaining data to fulfill the desired read size.  Vorbis uses truncated
packets as a normal mode of operation, and as such, decoders must
handle reading past the end of a packet as a typical mode of
operation. Any further read operations after an 'end-of-packet'
condition shall also return 'end-of-packet'.



\subsubsection{reading zero bits}

Reading a zero-bit-wide integer returns the value '0' and does not
increment the stream cursor.  Reading to the end of the packet (but
not past, such that an 'end-of-packet' condition has not triggered)
and then reading a zero bit integer shall succeed, returning 0, and
not trigger an end-of-packet condition.  Reading a zero-bit-wide
integer after a previous read sets 'end-of-packet' shall also fail
with 'end-of-packet'.






